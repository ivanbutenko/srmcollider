% Autor Tobias Erbsland.
% Veröffentlicht unter The GNU Free Documentation License.
% modifiziert von Hannes Röst
% Version 1.2
%
% A. DOKUMENTKLASSE
% ---------------------------------------------------------------------------
%

%
%  1. Definieren der Dokumentklasse.
%     Wir verwenden die KOMA-Script Klasse 'scr...' 
%
\documentclass[
  pdftex,           %PDFTex verwenden da wir ausschliesslich ein PDF erzeugen.
  a4paper,          %Wir verwenden A4 Papier.
  oneside,          %Einseitiger Druck.
  12pt,             %12 Punkte, besser geeignet für A4.
  halfparskip,      %Halbe Zeile Abstand zwischen Absätzen.
  %chapterprefix,    %Kapitel mit 'Kapitel' anschreiben.
  %headsepline,      %Linie nach Kopfzeile.
  %footsepline,      %Linie vor Fusszeile.
  bibtotocnumbered, %Literaturverzeichnis im Inhaltsverz. nummeriert einfügen.
  idxtotoc          %Index ins Inhaltsverzeichnis einfügen.
]{scrartcl}


%
%  2. Festlegen der Zeichencodierung des Dokuments und des Zeichensatzes.
%     Wir verwenden 'UTF-8' für das Dokument,
%     und die 'T1' codierung für die Schrift.
%
\usepackage[T1]{fontenc}
\usepackage[utf8]{inputenc}

%  3. Packet für die Index-Erstellung laden.
%     makeindex ist ein Programm zur Erstellung von Stichwortverzeichnissen für LaTeX-Dokumente
%
\usepackage{makeidx}

%
%  4. Neue deutsche Rechtschreibung mit 'ngerman'. Es kommt es zu einer Übersetzung im Dokument
%     (Inhaltsverzeichnis statt table of contents, Daten). Außerdem wird nach deutscher
%     Rechtschreibung getrennt.
\usepackage[ngerman, english]{babel}


%
%  5. Paket für Anführungszeichen laden.
%     Wir setzen den Stil auf 'swiss', und verwenden so die Schweizer Anführungszeichen.
%
%\usepackage[style=swiss]{csquotes}


%
%  6. Paket für erweiterte Tabelleneigenschaften.
%
\usepackage{array}

%
%  7. Paket um Grafiken im Dokument einbetten zu können.
%     Im PDF sind GIF, PNG, und PDF Grafiken möglich.
%
\usepackage{graphicx}

%
%  8. Pakete für mathematischen Textsatz.
%     siehe dazu auch http://www.andy-roberts.net/misc/latex/latextutorial10.html
\usepackage{amsmath}
\usepackage{amssymb}
\usepackage{dsfont}
\usepackage{mathtools}

%
%  9. Paket um Textteile drehen zu können.
%     \begin{turn}{Winkel} Text \end{turn}
\usepackage{rotating}

%
% 10. Paket für Farben an verschieden Stellen. 
%     Wir definieren zusätzliche benannte Farben.
%     Mit dem Befehl \pagecolor{Farbe} wird die Hintergrundfarbe, und mit dem 
%     Befehl \color{Farbe} die Textfarbe bestimmt. 
%     http://www.willemer.de/informatik/text/texcolor.htm
\usepackage{color}

%
% 11. Paket für spezielle PDF features.
%
%     pdftitle     Titel des PDF Dokuments.
%     pdfauthor    Autor des PDF Dokuments.
%     pdfsubject   Thema des PDF Dokuments.
%     pdfcreator   Erzeuger des PDF Dokuments.
%     pdfkeywords  Schlüsselwörter für das PDF.
%     pdfpagemode  Inhaltsverzeichnis anzeigen beim Öffnen
%     pdfdisplaydoctitle     Dokumenttitel statt Dateiname anzeigen.
%     pdflang      Sprache des Dokuments.
\usepackage[
  pdfpagemode=UseOutlines,
  pdfdisplaydoctitle=true,
  %pdflang=en,
  pdfauthor={Hannes Roest},
  %pdftitle={Hannes Roest},
  colorlinks=true,
  linkcolor=black,
  citecolor=black,
  filecolor=black,
  pagecolor=black,
  urlcolor=black
%  frenchlinks=true
]{hyperref}

%
% 12. Font wenn die Schrift nicht schön aussieht, verwende eine andere
%     siehe auch http://www.matthiaspospiech.de/latex/vorlagen/allgemein/preambel/fonts/
%\usepackage{mathptmx}
%\usepackage[scaled=.90]{helvet}
%\usepackage{courier}


%
% 13. Line Spacing
%     This might give better (and might give worse) results

%\usepackage{setspace}
%\singlespacing        %% 1-spacing (default)
%\onehalfspacing       %% 1,5-spacing
%\doublespacing        %% 2-spacing


%
% 14. Kopf und Fusszeilen
%

\usepackage{fancyhdr}
\pagestyle{fancy}

\fancyhf{} % aktuelle header and footer löschen
% Buchstabencodes für \fancyhead und \fancyfoot sind:
% E     Even page
% O     Odd page
% L     Left
% C     Center
% R     Right
% Können gemixt werden, also auch: \fancyhead[RO,RE]{irgendetwas} für 
%          "rechts oben auf geraden und ungeraden seiten"
% "Gerade" und "ungerade" Seite werden unterschieden für Bücher, wo man die
% Seitenzahl immer aussen haben möchte.
%\renewcommand{\chaptermark}[1]{\markboth{#1}{}}
\renewcommand{\sectionmark}[1]{\markright{#1}{}}
\fancyhead[L]{\thechapter \phantom{L}  \leftmark}
\fancyhead[C]{}
\fancyhead[R]{\thesection \phantom{L} \rightmark}
\fancyfoot[L]{}
\fancyfoot[C]{\thepage }
%Linie vor/nach Fuss/Kopfzeile
\renewcommand{\headrulewidth}{0.0pt} 
\renewcommand{\footrulewidth}{0.0pt}
%\addtolength{\headheight}{0.5pt} % platz machen für die linie
%\addtolength{\footheight}{0.5pt} % platz machen für die linie
\fancypagestyle{plain}{%
       \fancyhead{} % kopfzeilen auf leeren seiten loswerden
       \renewcommand{\headrulewidth}{0pt} % ... und auch die linie
       \renewcommand{\footrulewidth}{0pt} % ... und auch die linie
}

%%
%% 15. Schöne Boxen
%%
%\usepackage{fancybox}

%% File Extensions of Graphics %%%%%%%%%%%%%%%%%%%%%%%%%%%%%%
%% ==> This enables you to omit the file extension of a graphic.
%% ==> "\includegraphics{title.eps}" becomes "\includegraphics{title}".
%% ==> If you create 2 graphics with same content (but different file types)
%% ==> "title.eps" and "title.pdf", only the file processable by
%% ==> your compiler will be used.
%% ==> pdfLaTeX uses "title.pdf". LaTeX uses "title.eps".
\DeclareGraphicsExtensions{.pdf,.jpg,.png}

%% Martina's preferences %%%%%%%%%%%%%%%%%%%%%%%%%%%%%%%%%%%%
%\setlength{\parskip}{6pt} %Legt den Abstand zwischen den nachfolgenden Absätzen fest. 
\usepackage{url} %besser mit hypperref
%\usepackage[paren, plain]{fancyref}
%\newcommand{\species}[1]{\textsl{#1}}  %%formating species names
%\setcounter{fignumdepth}{1} %%How can figurs be numbered independently of the chapter


%% Other Packages %%%%%%%%%%%%%%%%%%%%%%%%%%%%%%%%%%%%%%%%%%%
%\usepackage{a4wide} %%Smaller margins = more text per page.
%\usepackage{longtable} %%For tables, that exceed one page


%%Attention: the correct dash is the following: -


\usepackage{gensymb}
\usepackage{hyphenat}
\usepackage{subfig}

\usepackage{booktabs} %make lines in tables (midrule, toprule etc)
\newcommand*\maketablespace{ \addlinespace[15pt] }

%\setcounter{tocdepth}{3}
\begin{document}

\author{Hannes Röst}
\title{Lab Notebook SRM clashes}
\maketitle

\tableofcontents

\section{Large background, 300 to 2000 Da}
I worked now in Pedros idea with very large Q1 windows (25 Da) and very
small Q1 windows (1 or 10 ppm). I had to introduce some more code to be
able to cope with the ppm story. I also had to check that the peptides
chosen were unique (even though those in the background were not)

We chose to search 2+ parent and 1+ fragment ion against a background of
2+ parent (1+,2+ fragment) and 3+ parent (1+,2+ fragment). SSRCalcRange
(window size) was 2 arbitrary units which is about 1 minute, depending
on the coloumn.

\subsubsection{25 Da window}

\begin{verbatim}
check all four charge states [False] vs all four charge 
states [True] with thresholds of SSRCalc 2.0, Q1 25.0 (Th), Q3 10.0 (PPM) and 
a range of 300 - 2000 Da for the q3 transitions. 
Ratio of non-unique peptides is 0.280395401014
\end{verbatim}

%Figure~\ref{25Da_10ppm} we have a yeast dataset, a complete digested proteome from
all proteins found by Ensembl. Only 2+ parent ions and 1+ fragment ions of
unique peptides were used for the analysis wheras a background of all peptides
(also non-unique) and 2+ parent (1+,2+ fragment) and 3+ parent (1+,2+ fragment)
was used.  We used windows of 25 Da for Q1, 10 ppm for Q3 and 2 SSRCalc units.
We see that accuracy from 10 ppm to 1 ppm will help only in very few cases on
Q3 level.

%\begin{figure}
%
%\center
%\subfloat[1][1]{\includegraphics[width=0.6\textwidth, angle=-90]{/home/hroest/srm_clashes/results/pedro/yeast_True_False_20_250_100ppm_range300to2000.pdf}} \\
%\subfloat[2][2]{\includegraphics[width=0.6\textwidth, angle=-90]{/home/hroest/srm_clashes/results/pedro/yeast_True_False_20_250_100ppm_range300to2000_q3distr_ppm.pdf}}
%
%\caption{
%Searched 2+ parent and 1+ fragment ion against a background of 2+ parent (1+,2+ fragment) and 3+ parent (1+,2+ fragment).
%SSRCalcRange was 2 arbitrary units, Q1 25 Da and Q3 10 ppm
%}
%\label{25Da_10ppm}
%\end{figure}
%\begin{figure}
%\subfloat[1][Zoom in]{\includegraphics[width=0.6\textwidth,angle=-90]{/home/hroest/srm_clashes/results/pedro/yeast_True_False_20_250_100ppm_range300to2000_q3distr_ppm_small_window.pdf}} 
%\subfloat[2][Th instead of ppm]{\includegraphics[width=0.6\textwidth, angle=-90]{/home/hroest/srm_clashes/results/pedro/yeast_True_False_20_250_100ppm_range300to2000_q3distr.pdf}}\\
%\caption{Top: With Th instead of ppm}
%\label{some_label}
%\end{figure}


%\subsubsection{9 Da window}
%Figure~\ref{9Da_10ppm} shows the same with 9 Da window (10 ppm on Q3)
%
%\begin{figure}
%\center
%\subfloat[][Unique transitions per peptide]{\includegraphics[width=0.6\textwidth,angle=-90]{/home/hroest/srm_clashes/results/pedro/yeast_True_False_20_90_100ppm_range300to2000.pdf} }  \\
%\subfloat[][Q3 distribution]{\includegraphics[width=0.6\textwidth,angle=-90]{/home/hroest/srm_clashes/results/pedro/yeast_True_False_20_90_100ppm_range300to2000_q3distr_ppm.pdf}}
%%
%\caption{
%Searched 2+ parent and 1+ fragment ion against a background of 2+ parent (1+,2+ fragment) and 3+ parent (1+,2+ fragment).
%SSRCalcRange was 2 arbitrary units, Q1 9 Da and Q3 10 ppm
%}
%\label{9Da_10ppm}
%\end{figure}
%
%
%\subsubsection{1 Da window}
%Figure~\ref{1Da_10ppm} shows the same with 1 Da window (10 ppm on Q3)
%
%\begin{figure}
%\center
%\subfloat[][Unique transitions per peptide]{\includegraphics[width=0.6\textwidth,angle=-90]{/home/hroest/srm_clashes/results/pedro/yeast_True_False_20_10_100ppm_range300to2000.pdf} }  \\
%\subfloat[][Unique transitions per peptide]{\includegraphics[width=0.6\textwidth,angle=-90]{/home/hroest/srm_clashes/results/pedro/yeast_True_False_20_10_100ppm_range300to2000_cum.pdf} }  
%%\subfloat[][Q3 distribution]{\includegraphics[width=0.6\textwidth,angle=-90]{/home/hroest/srm_clashes/results/pedro/yeast_True_False_20_10_100ppm_range300to2000_q3distr_ppm.pdf}}
%%
%\caption{
%Searched 2+ parent and 1+ fragment ion against a background of 2+ parent (1+,2+ fragment) and 3+ parent (1+,2+ fragment).
%SSRCalcRange was 2 arbitrary units, Q1 9 Da and Q3 10 ppm
%}
%\label{1Da_10ppm}
%\end{figure}
%
%
%\subsubsection{0.2 Da window}
%Figure~\ref{02Da_1Da} shows 0.2 Da on Q1 and 1 Da on Q3.
%Here we have a range of 300 Da to 1500 Da only.
%
%\begin{figure}
%
%\center
%%\subfloat[1][1]{\includegraphics[width=0.6\textwidth, angle=-90]{/home/hroest/srm_clashes/results/yeast/yeast_True_False_20210_range300to1500_lendist.pdf}} \\
%\subfloat[2][2]{\includegraphics[width=0.6\textwidth, angle=-90]{/home/hroest/srm_clashes/results/yeast/yeast_True_False_20210_range300to1500.pdf}}
%
%\caption{
%Searched 2+ parent and 1+ fragment ion against a background of 2+ parent (1+,2+ fragment) and 3+ parent (1+,2+ fragment).
%SSRCalcRange was 2 arbitrary units, Massrange for Q3 300-1500 Da, Q1 window 0.2 Da
%}
%\label{02Da_1Da}
%\end{figure}
%
%
%
%
%
%\clearpage
%
\section{Large background, 300 to minus 300 Da}
\subsection{25 Da, 10 ppm }

\begin{verbatim}
Experiment Type:
    check all four charge states [False] vs all four charge states [True] with
    thresholds of SSRCalc 2.0, Q1 25.0 (Th), Q3 10.0 (PPM) and a range of 300 to -300
    Da for the q3 transitions.  
    Ratio of non-unique peptides is 0.268012370107
\end{verbatim}


\begin{figure}

\center
\subfloat[1][1]{\includegraphics[width=0.6\textwidth, angle=-90]{/home/hroest/srm_clashes/results/pedro/yeast_True_False_20_250_100ppm_range300tomin300.pdf}} \\
\subfloat[1][1]{\includegraphics[width=0.6\textwidth, angle=-90]{/home/hroest/srm_clashes/results/pedro/yeast_True_False_20_250_100ppm_range300tomin300_cum.pdf}} 
%\subfloat[2][2]{\includegraphics[width=0.6\textwidth, angle=-90]{/home/hroest/srm_clashes/results/pedro/yeast_True_False_20_250_100ppm_range300tomin300_q3distr_ppm.pdf}}

\caption{
Searched 2+ parent and 1+ fragment ion against a background of 2+ parent (1+,2+ fragment) and 3+ parent (1+,2+ fragment).
SSRCalcRange was 2 arbitrary units, Q1 25 Da and Q3 10 ppm
}
\label{fig:2.25Da_10ppm}
\end{figure}
\begin{figure}

\center
\subfloat[2][2]{\includegraphics[width=0.6\textwidth, angle=-90]{/home/hroest/srm_clashes/results/pedro/yeast_True_False_20_250_100ppm_range300tomin300_q3distr_ppm.pdf}}

\caption{
Searched 2+ parent and 1+ fragment ion against a background of 2+ parent (1+,2+ fragment) and 3+ parent (1+,2+ fragment).
SSRCalcRange was 2 arbitrary units, Q1 25 Da and Q3 10 ppm
}
\label{fig:2.25Da_10ppm_q3dist}
\end{figure}



\clearpage
\section{Peptides between 400 and 1200 Da}

%
%\begin{verbatim}
%
%drop table tmptbl;
%create temporary table tmptbl as
%select count(*) as occ , srmPeptides_yeast.peptide_key from hroest.srmPeptides_yeast 
%inner join ddb.peptide on srmPeptides_yeast.peptide_key = peptide.id 
%inner join ddb.peptideOrganism on peptideOrganism.peptide_key = peptide.id
%where q1 > 400 and q1 < 1200 
%and genome_occurence = 1
%group by srmPeptides_yeast.peptide_key
%;
%alter table tmptbl add index(peptide_key);
%
%select count(*) from tmptbl
%#inner join ddb.protPepLink on protPepLink.peptide_key = tmptbl.peptide_key
%#group by protein_key
%;
%
%select count(*) from tmptbl
%#inner join ddb.protPepLink on protPepLink.peptide_key = tmptbl.peptide_key
%where occ = 2
%#group by protein_key
%;
%
%
%#
%##
%#
%# -- check whether we really filtered out non-unique peptides
%create temporary table unique_yeast_proteins as
%select protein_key from ddb.peptide 
%#select count(*) from ddb.peptide 
%inner join ddb.protPepLink on protPepLink.peptide_key = peptide.id
%inner join ddb.peptideOrganism on peptideOrganism.peptide_key = peptide.id
%where experiment_key = 3120 
%and genome_occurence = 1
%group by protein_key
%;
%
%drop table non_unique_yeast_proteins;
%create temporary table non_unique_yeast_proteins as
%select distinct protein_key from ddb.peptide 
%inner join ddb.protPepLink on protPepLink.peptide_key = peptide.id
%where experiment_key = 3120 
%and protein_key not in (select * from unique_yeast_proteins)
%;
%
%#see how many peptides each of the non unique proteins has
%select protPepLink.protein_key, count(*) from
%non_unique_yeast_proteins 
%inner join ddb.protPepLink on protPepLink.protein_key = non_unique_yeast_proteins.protein_key
%group by protPepLink.protein_key;
%
%
%#check one protein
%select genome_occurence, sequence from ddb.peptide
%inner join ddb.protPepLink on protPepLink.peptide_key = peptide.id
%inner join ddb.peptideOrganism on peptideOrganism.peptide_key = peptide.id
%where protein_key = 2818530;
%
%#check one sequence
%select * from ddb.peptide 
%inner join ddb.protPepLink on protPepLink.peptide_key = peptide.id
%where sequence = 'SLEDNETEIK' and experiment_key = 3120;
%
%
%
%\end{verbatim}
%

See table~\ref{tab:yeast_pepprot} for the peptide and protein analysis. It
shows how many peptides and proteins are visible in a window of 400 to 1200 Da
and how many of those have 2+ and 3+ charged parent ions in that window. I
calculated it for all peptides and for the unique peptides only.

\begin{table}[h]

\centering
\caption[Yeast peptide and protein distribution.]
{\textbf{Yeast peptide and protein distribution.}
The number of proteins and peptides in yeast. Shown are all, those which have
parent ions between 400 and 1200 Da and finally those which additionally also
have both 2+ and 3+ parent ions in that range. \newline
The top table show all peptides, the bottom table shows only those which occur
once in the genome.
}
\label{tab:yeast_pepprot}

\begin{tabular}{ l c c }
\maketablespace
Description & Peptides  & Proteins \\
\toprule
All & 137\,161 & 6602 \\
400-1200 & 130\,734 & 6593 \\
400-1200, 2+ and 3+ parent ions & \phantom{1}62\,663 & 6422\\
\toprule
All & 134\,412 & 6437 \\
400-1200 & 128\,094 & 6419 \\
400-1200, 2+ and 3+ parent ions & \phantom{1}61\,530 & 6180\\

\end{tabular}
\end{table}

All 128\,094 peptides in this range have 1\,690\,316 transitions.

All the following data was created with these parameters: 
Searched 2+ parent and 1+ fragment ion against a background of 2+ parent (1+,2+
fragment) and 3+ parent (1+,2+ fragment). The background was the ensembl yeast
genome release, release 57.

The SSRCalc window was 2 arbitrary units, Q1 variable and Q3 was 10 ppm.

Only ions from peptides with a molecular mass between 800\,Da and 5000\,Da 
and transitions between 400 and 1200\,Th were included. 
\subsection{50 Da}

%Non unique transitions are 0.51060038478~\% or 863\,076
Nonunique / Total transitions : 863076.0 / 1690316.0 = 0.51060038478

Percentage of collisions below 1 ppm: 74.79~\%


See Figure~\ref{fig:400range.50Da_10ppm} and Figure~\ref{fig:400range.50Da_10ppm_q3dist}.

\begin{figure}

\center
\subfloat[1][Unique transition distribution]{\includegraphics[width=0.6\textwidth, angle=-90]{/home/hroest/srm_clashes/results/pedro/yeast_True_False_20_500_100ppm_range400to1200.pdf}} \\
\subfloat[1][Cumulative unique transition distribution]{\includegraphics[width=0.6\textwidth, angle=-90]{/home/hroest/srm_clashes/results/pedro/yeast_True_False_20_500_100ppm_range400to1200_cum.pdf}} 

\caption{ \textbf{Percentage unique transitions in the tryptic yeast peptidome.}
The number of peptides is plotted against the percentage of unique transitions
per peptide (top). Below, a cumulative distribution is shown.
}
\label{fig:400range.50Da_10ppm}
\end{figure}
\begin{figure}

\center
\subfloat[2][Q3 distance distribution]{\includegraphics[width=0.6\textwidth, angle=-90]{/home/hroest/srm_clashes/results/pedro/yeast_True_False_20_500_100ppm_range400to1200_q3distr_ppm.pdf}}

\caption{
Searched 2+ parent and 1+ fragment ion against a background of 2+ parent (1+,2+ fragment) and 3+ parent (1+,2+ fragment).
SSRCalcRange was 2 arbitrary units, Q1 25 Da and Q3 10 ppm
}
\label{fig:400range.50Da_10ppm_q3dist}
\end{figure}

\subsection{25 Da}

Nonunique / Total transitions : 602972.0 / 1690316.0 = 0.356721465099
Percentage of collisions below 1 ppm: 73.63~\%


See Figure~\ref{fig:400range.25Da_10ppm} and Figure~\ref{fig:400range.25Da_10ppm_q3dist}.

\begin{figure}

\center
\subfloat[1][Unique transition distribution]{\includegraphics[width=0.6\textwidth, angle=-90]{/home/hroest/srm_clashes/results/pedro/yeast_True_False_20_250_100ppm_range400to1200.pdf}} \\
\subfloat[1][Cumulative unique transition distribution]{\includegraphics[width=0.6\textwidth, angle=-90]{/home/hroest/srm_clashes/results/pedro/yeast_True_False_20_250_100ppm_range400to1200_cum.pdf}} 

\caption{ \textbf{Percentage unique transitions in the tryptic yeast peptidome.}
The number of peptides is plotted against the percentage of unique transitions
per peptide (top). Below, a cumulative distribution is shown.
}
\label{fig:400range.25Da_10ppm}
\end{figure}
\begin{figure}

\center
\subfloat[2][Q3 distance distribution]{\includegraphics[width=0.6\textwidth, angle=-90]{/home/hroest/srm_clashes/results/pedro/yeast_True_False_20_250_100ppm_range400to1200_q3distr_ppm.pdf}}

\caption{
Searched 2+ parent and 1+ fragment ion against a background of 2+ parent (1+,2+ fragment) and 3+ parent (1+,2+ fragment).
SSRCalcRange was 2 arbitrary units, Q1 25 Da and Q3 10 ppm
}
\label{fig:400range.25Da_10ppm_q3dist}
\end{figure}


\subsection{15 Da}

Nonunique / Total transitions : 448789 / 1690316 = 0.265505976397
Percentage of collisions below 1 ppm: 16.59~\%


See Figure~\ref{fig:400range.15Da_10ppm} and Figure~\ref{fig:400range.15Da_10ppm_q3dist}.

\begin{figure}

\center
\subfloat[1][Unique transition distribution]{\includegraphics[width=0.6\textwidth, angle=-90]{/home/hroest/srm_clashes/results/pedro/yeast_True_False_20_150_100ppm_range400to1200.pdf}} \\
\subfloat[1][Cumulative unique transition distribution]{\includegraphics[width=0.6\textwidth, angle=-90]{/home/hroest/srm_clashes/results/pedro/yeast_True_False_20_150_100ppm_range400to1200_cum.pdf}} 

\caption{ \textbf{Percentage unique transitions in the tryptic yeast peptidome.}
The number of peptides is plotted against the percentage of unique transitions
per peptide (top). Below, a cumulative distribution is shown.
}
\label{fig:400range.15Da_10ppm}
\end{figure}
\begin{figure}

\center
\subfloat[2][Q3 distance distribution]{\includegraphics[width=0.6\textwidth, angle=-90]{/home/hroest/srm_clashes/results/pedro/yeast_True_False_20_150_100ppm_range400to1200_q3distr_ppm.pdf}}

\caption{
Searched 2+ parent and 1+ fragment ion against a background of 2+ parent (1+,2+ fragment) and 3+ parent (1+,2+ fragment).
SSRCalcRange was 2 arbitrary units, Q1 15 Da and Q3 10 ppm
}
\label{fig:400range.15Da_10ppm_q3dist}
\end{figure}


\subsection{9 Da}

Nonunique / Total transitions : 310193.0 / 1690316.0 = 0.18351184039

Percentage of collisions below 1 ppm: 74.29~\%


See Figure~\ref{fig:400range.9Da_10ppm} and Figure~\ref{fig:400range.9Da_10ppm_q3dist}.

\begin{figure}

\center
\subfloat[1][Unique transition distribution]{\includegraphics[width=0.6\textwidth, angle=-90]{/home/hroest/srm_clashes/results/pedro/yeast_True_False_20_90_100ppm_range400to1200.pdf}} \\
\subfloat[1][Cumulative unique transition distribution]{\includegraphics[width=0.6\textwidth, angle=-90]{/home/hroest/srm_clashes/results/pedro/yeast_True_False_20_90_100ppm_range400to1200_cum.pdf}} 

\caption{ \textbf{Percentage unique transitions in the tryptic yeast peptidome.}
The number of peptides is plotted against the percentage of unique transitions
per peptide (top). Below, a cumulative distribution is shown.
}
\label{fig:400range.9Da_10ppm}
\end{figure}
\begin{figure}

\center
\subfloat[2][Q3 distance distribution]{\includegraphics[width=0.6\textwidth, angle=-90]{/home/hroest/srm_clashes/results/pedro/yeast_True_False_20_90_100ppm_range400to1200_q3distr_ppm.pdf}}

\caption{
Searched 2+ parent and 1+ fragment ion against a background of 2+ parent (1+,2+ fragment) and 3+ parent (1+,2+ fragment).
SSRCalcRange was 2 arbitrary units, Q1 9 Da and Q3 10 ppm
}
\label{fig:400range.9Da_10ppm_q3dist}
\end{figure}


\subsection{1 Da}

Nonunique / Total transitions : 86612.0 / 1690316.0 = 0.0512401231486

Percentage of collisions below 1 ppm: 79.77~\%


See Figure~\ref{fig:400range.1Da_10ppm} and Figure~\ref{fig:400range.1Da_10ppm_q3dist}.

\begin{figure}

\center
\subfloat[1][Unique transition distribution]{\includegraphics[width=0.6\textwidth, angle=-90]{/home/hroest/srm_clashes/results/pedro/yeast_True_False_20_10_100ppm_range400to1200.pdf}} \\
\subfloat[1][Cumulative unique transition distribution]{\includegraphics[width=0.6\textwidth, angle=-90]{/home/hroest/srm_clashes/results/pedro/yeast_True_False_20_10_100ppm_range400to1200_cum.pdf}} 

\caption{ \textbf{Percentage unique transitions in the tryptic yeast peptidome.}
The number of peptides is plotted against the percentage of unique transitions
per peptide (top). Below, a cumulative distribution is shown.
}
\label{fig:400range.1Da_10ppm}
\end{figure}
\begin{figure}

\center
\subfloat[2][Q3 distance distribution]{\includegraphics[width=0.6\textwidth, angle=-90]{/home/hroest/srm_clashes/results/pedro/yeast_True_False_20_10_100ppm_range400to1200_q3distr_ppm.pdf}}

\caption{
Searched 2+ parent and 1+ fragment ion against a background of 2+ parent (1+,2+ fragment) and 3+ parent (1+,2+ fragment).
SSRCalcRange was 2 arbitrary units, Q1 1 Da and Q3 10 ppm
}
\label{fig:400range.1Da_10ppm_q3dist}
\end{figure}


\subsection{Traditional SRM: 0.7 Da, 1 Da}

This is the comparison to traditional SRM with Q1 of 0.7 Da and Q3 of 1 Da.

Nonunique / Total transitions : 418245.0 / 1690316.0 = 0.247435982384

Percentage of collisions below 1 ppm: 20.04~\%


See Figure~\ref{fig:400range.07Da_1Da} and Figure~\ref{fig:400range.07Da_1Da_q3dist}.

\begin{figure}

\center
\subfloat[1][Unique transition distribution]{\includegraphics[width=0.6\textwidth, angle=-90]{/home/hroest/srm_clashes/results/pedro/yeast_True_False_20_7_10_range400to1200.pdf}} \\
\subfloat[1][Cumulative unique transition distribution]{\includegraphics[width=0.6\textwidth, angle=-90]{/home/hroest/srm_clashes/results/pedro/yeast_True_False_20_7_10_range400to1200_cum.pdf}} 

\caption{ \textbf{Percentage unique transitions in the tryptic yeast peptidome.}
The number of peptides is plotted against the percentage of unique transitions
per peptide (top). Below, a cumulative distribution is shown.
}
\label{fig:400range.07Da_1Da}
\end{figure}
\begin{figure}

\center
\subfloat[2][Q3 distance distribution]{\includegraphics[width=0.6\textwidth, angle=-90]{/home/hroest/srm_clashes/results/pedro/yeast_True_False_20_7_10_range400to1200_q3distr_ppm.pdf}}

\caption{
Searched 2+ parent and 1+ fragment ion against a background of 2+ parent (1+,2+ fragment) and 3+ parent (1+,2+ fragment).
SSRCalcRange was 2 arbitrary units, Q1 0.7 Da and Q3 1 Da
}
\label{fig:400range.07Da_1Da_q3dist}
\end{figure}


\subsection{Comparison}




\begin{table}[h]

\centering
\caption[]
{\textbf{Percentage of non-unique transitions}
Comparison of the methods explored above using the percentage of non-unique
transitions of the total number of transitions.
}
\label{tab:comp_nonunique}

\begin{tabular}{ l  c }
\maketablespace
Method &Non-unique transitions / \%  \\
\toprule
SRM (0.7 Da, 1 Da) & 24.7 \\
\midrule
1 Da, 10 ppm&  \phantom{1}5.1 \\
9 Da, 10 ppm&  18.3\\
15 Da, 10 ppm& 26.5\\
25 Da, 10 ppm& 35.7\\
50 Da, 10 ppm& 51.0\\

\end{tabular}
\end{table}


See Figure~\ref{fig:400range.comp} and Figure~\ref{fig:400range.comp_cum}.

\begin{figure}

\center
\subfloat[1][Traditional: 0.7 Da, 0.7 Da]{\includegraphics[width=0.3\textwidth, angle=-90]{/home/hroest/srm_clashes/results/pedro/yeast_True_False_20_7_10_range400to1200_comp.pdf}} 
\subfloat[1][1Da, 10ppm]{\includegraphics[width=0.3\textwidth, angle=-90]{/home/hroest/srm_clashes/results/pedro/yeast_True_False_20_10_100ppm_range400to1200_comp.pdf}} \\
\subfloat[1][9Da, 10ppm]{\includegraphics[width=0.3\textwidth, angle=-90]{/home/hroest/srm_clashes/results/pedro/yeast_True_False_20_90_100ppm_range400to1200_comp.pdf}} 
\subfloat[1][15Da, 10ppm]{\includegraphics[width=0.3\textwidth, angle=-90]{/home/hroest/srm_clashes/results/pedro/yeast_True_False_20_150_100ppm_range400to1200_comp.pdf}} \\
\subfloat[1][25Da, 10ppm]{\includegraphics[width=0.3\textwidth, angle=-90]{/home/hroest/srm_clashes/results/pedro/yeast_True_False_20_250_100ppm_range400to1200_comp.pdf}} 
\subfloat[1][50Da, 10ppm]{\includegraphics[width=0.3\textwidth, angle=-90]{/home/hroest/srm_clashes/results/pedro/yeast_True_False_20_500_100ppm_range400to1200_comp.pdf}} \\

\caption{ \textbf{Percentage unique transitions in the tryptic yeast peptidome.}
Comparison of the histograms.  }
\label{fig:400range.comp}
\end{figure}
\begin{figure}

\center
\subfloat[1][Traditional: 0.7 Da, 0.7 Da]{\includegraphics[width=0.3\textwidth, angle=-90]{/home/hroest/srm_clashes/results/pedro/yeast_True_False_20_7_10_range400to1200_cum.pdf}} 
\subfloat[1][1Da, 10ppm]{\includegraphics[width=0.3\textwidth, angle=-90]{/home/hroest/srm_clashes/results/pedro/yeast_True_False_20_10_100ppm_range400to1200_cum.pdf}} \\
\subfloat[1][9Da, 10ppm]{\includegraphics[width=0.3\textwidth, angle=-90]{/home/hroest/srm_clashes/results/pedro/yeast_True_False_20_90_100ppm_range400to1200_cum.pdf}} 
\subfloat[1][15Da, 10ppm]{\includegraphics[width=0.3\textwidth, angle=-90]{/home/hroest/srm_clashes/results/pedro/yeast_True_False_20_150_100ppm_range400to1200_cum.pdf}} \\
\subfloat[1][25Da, 10ppm]{\includegraphics[width=0.3\textwidth, angle=-90]{/home/hroest/srm_clashes/results/pedro/yeast_True_False_20_250_100ppm_range400to1200_cum.pdf}} 
\subfloat[1][50Da, 10ppm]{\includegraphics[width=0.3\textwidth, angle=-90]{/home/hroest/srm_clashes/results/pedro/yeast_True_False_20_500_100ppm_range400to1200_cum.pdf}} \\

\caption{ \textbf{Percentage unique transitions in the tryptic yeast peptidome.}
Comparison of the cumulative distributions.  }
\label{fig:400range.comp_cum}
\end{figure}

\begin{figure}

\center
\includegraphics[width=0.7\textwidth, angle=-90]{/home/hroest/srm_clashes/results/multiple_csv/cumm_plot.pdf}

\caption{ \textbf{Percentage unique transitions in the tryptic yeast peptidome.}
Comparison of the cumulative distributions using different Q1 window sizes and
10 ppm accuracy against the SRM standard: 0.7 Da Q1 and 1.0 Da Q3 window.}
\label{fig:400range.comp_cum_all}
\end{figure}




\end{document}
