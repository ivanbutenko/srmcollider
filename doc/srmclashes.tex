% Autor Tobias Erbsland.
% Veröffentlicht unter The GNU Free Documentation License.
% modifiziert von Hannes Röst
% Version 1.2
%
% A. DOKUMENTKLASSE
% ---------------------------------------------------------------------------
%

%
%  1. Definieren der Dokumentklasse.
%     Wir verwenden die KOMA-Script Klasse 'scr...' 
%
\documentclass[
  pdftex,           %PDFTex verwenden da wir ausschliesslich ein PDF erzeugen.
  a4paper,          %Wir verwenden A4 Papier.
  oneside,          %Einseitiger Druck.
  12pt,             %12 Punkte, besser geeignet für A4.
  halfparskip,      %Halbe Zeile Abstand zwischen Absätzen.
  %chapterprefix,    %Kapitel mit 'Kapitel' anschreiben.
  %headsepline,      %Linie nach Kopfzeile.
  %footsepline,      %Linie vor Fusszeile.
  bibtotocnumbered, %Literaturverzeichnis im Inhaltsverz. nummeriert einfügen.
  idxtotoc          %Index ins Inhaltsverzeichnis einfügen.
]{scrartcl}


%
%  2. Festlegen der Zeichencodierung des Dokuments und des Zeichensatzes.
%     Wir verwenden 'UTF-8' für das Dokument,
%     und die 'T1' codierung für die Schrift.
%
\usepackage[T1]{fontenc}
\usepackage[utf8]{inputenc}

%  3. Packet für die Index-Erstellung laden.
%     makeindex ist ein Programm zur Erstellung von Stichwortverzeichnissen für LaTeX-Dokumente
%
\usepackage{makeidx}

%
%  4. Neue deutsche Rechtschreibung mit 'ngerman'. Es kommt es zu einer Übersetzung im Dokument
%     (Inhaltsverzeichnis statt table of contents, Daten). Außerdem wird nach deutscher
%     Rechtschreibung getrennt.
\usepackage[ngerman, english]{babel}


%
%  5. Paket für Anführungszeichen laden.
%     Wir setzen den Stil auf 'swiss', und verwenden so die Schweizer Anführungszeichen.
%
%\usepackage[style=swiss]{csquotes}


%
%  6. Paket für erweiterte Tabelleneigenschaften.
%
\usepackage{array}

%
%  7. Paket um Grafiken im Dokument einbetten zu können.
%     Im PDF sind GIF, PNG, und PDF Grafiken möglich.
%
\usepackage{graphicx}

%
%  8. Pakete für mathematischen Textsatz.
%     siehe dazu auch http://www.andy-roberts.net/misc/latex/latextutorial10.html
\usepackage{amsmath}
\usepackage{amssymb}
\usepackage{dsfont}
\usepackage{mathtools}

%
%  9. Paket um Textteile drehen zu können.
%     \begin{turn}{Winkel} Text \end{turn}
\usepackage{rotating}

%
% 10. Paket für Farben an verschieden Stellen. 
%     Wir definieren zusätzliche benannte Farben.
%     Mit dem Befehl \pagecolor{Farbe} wird die Hintergrundfarbe, und mit dem 
%     Befehl \color{Farbe} die Textfarbe bestimmt. 
%     http://www.willemer.de/informatik/text/texcolor.htm
\usepackage{color}

%
% 11. Paket für spezielle PDF features.
%
%     pdftitle     Titel des PDF Dokuments.
%     pdfauthor    Autor des PDF Dokuments.
%     pdfsubject   Thema des PDF Dokuments.
%     pdfcreator   Erzeuger des PDF Dokuments.
%     pdfkeywords  Schlüsselwörter für das PDF.
%     pdfpagemode  Inhaltsverzeichnis anzeigen beim Öffnen
%     pdfdisplaydoctitle     Dokumenttitel statt Dateiname anzeigen.
%     pdflang      Sprache des Dokuments.
\usepackage[
  pdfpagemode=UseOutlines,
  pdfdisplaydoctitle=true,
  %pdflang=en,
  pdfauthor={Hannes Roest},
  %pdftitle={Hannes Roest},
  colorlinks=true,
  linkcolor=black,
  citecolor=black,
  filecolor=black,
  pagecolor=black,
  urlcolor=black
%  frenchlinks=true
]{hyperref}

%
% 12. Font wenn die Schrift nicht schön aussieht, verwende eine andere
%     siehe auch http://www.matthiaspospiech.de/latex/vorlagen/allgemein/preambel/fonts/
%\usepackage{mathptmx}
%\usepackage[scaled=.90]{helvet}
%\usepackage{courier}


%
% 13. Line Spacing
%     This might give better (and might give worse) results

%\usepackage{setspace}
%\singlespacing        %% 1-spacing (default)
%\onehalfspacing       %% 1,5-spacing
%\doublespacing        %% 2-spacing


%
% 14. Kopf und Fusszeilen
%

\usepackage{fancyhdr}
\pagestyle{fancy}

\fancyhf{} % aktuelle header and footer löschen
% Buchstabencodes für \fancyhead und \fancyfoot sind:
% E     Even page
% O     Odd page
% L     Left
% C     Center
% R     Right
% Können gemixt werden, also auch: \fancyhead[RO,RE]{irgendetwas} für 
%          "rechts oben auf geraden und ungeraden seiten"
% "Gerade" und "ungerade" Seite werden unterschieden für Bücher, wo man die
% Seitenzahl immer aussen haben möchte.
%\renewcommand{\chaptermark}[1]{\markboth{#1}{}}
\renewcommand{\sectionmark}[1]{\markright{#1}{}}
\fancyhead[L]{\thechapter \phantom{L}  \leftmark}
\fancyhead[C]{}
\fancyhead[R]{\thesection \phantom{L} \rightmark}
\fancyfoot[L]{}
\fancyfoot[C]{\thepage }
%Linie vor/nach Fuss/Kopfzeile
\renewcommand{\headrulewidth}{0.0pt} 
\renewcommand{\footrulewidth}{0.0pt}
%\addtolength{\headheight}{0.5pt} % platz machen für die linie
%\addtolength{\footheight}{0.5pt} % platz machen für die linie
\fancypagestyle{plain}{%
       \fancyhead{} % kopfzeilen auf leeren seiten loswerden
       \renewcommand{\headrulewidth}{0pt} % ... und auch die linie
       \renewcommand{\footrulewidth}{0pt} % ... und auch die linie
}

%%
%% 15. Schöne Boxen
%%
%\usepackage{fancybox}

%% File Extensions of Graphics %%%%%%%%%%%%%%%%%%%%%%%%%%%%%%
%% ==> This enables you to omit the file extension of a graphic.
%% ==> "\includegraphics{title.eps}" becomes "\includegraphics{title}".
%% ==> If you create 2 graphics with same content (but different file types)
%% ==> "title.eps" and "title.pdf", only the file processable by
%% ==> your compiler will be used.
%% ==> pdfLaTeX uses "title.pdf". LaTeX uses "title.eps".
\DeclareGraphicsExtensions{.pdf,.jpg,.png}

%% Martina's preferences %%%%%%%%%%%%%%%%%%%%%%%%%%%%%%%%%%%%
%\setlength{\parskip}{6pt} %Legt den Abstand zwischen den nachfolgenden Absätzen fest. 
\usepackage{url} %besser mit hypperref
%\usepackage[paren, plain]{fancyref}
%\newcommand{\species}[1]{\textsl{#1}}  %%formating species names
%\setcounter{fignumdepth}{1} %%How can figurs be numbered independently of the chapter


%% Other Packages %%%%%%%%%%%%%%%%%%%%%%%%%%%%%%%%%%%%%%%%%%%
%\usepackage{a4wide} %%Smaller margins = more text per page.
%\usepackage{longtable} %%For tables, that exceed one page


%%Attention: the correct dash is the following: -


\usepackage{gensymb}
\usepackage{hyphenat}
\usepackage{subfig}

%\setcounter{tocdepth}{3}
\begin{document}

\author{Hannes Röst}
\title{Lab Notebook}
\maketitle

\tableofcontents

\section{Human}

\subsection{Figures}


For all figures:
Q3 range = 1.0 Da
On the left side, the distribution by peptide length is shown, on the right side the overall distribution of percent unique transitions.
From top to bottom: 
\begin{itemize}
\item Massrange for Q3 300-1500 Da, Q1 window 0.2 Da
\item Massrange for Q3 0-10000  Da, Q1 window 0.7 Da
\item Massrange for Q3 300-1500 Da, Q1 window 0.7 Da
\end{itemize}

\begin{figure}

\subfloat[][1]{\includegraphics[width=0.35\textwidth, angle=-90]{/home/hroest/srm_clashes/results/True_False_20210_range300to1500_lendist.pdf}}
\subfloat[][2]{\includegraphics[width=0.35\textwidth, angle=-90]{/home/hroest/srm_clashes/results/True_False_20210_range300to1500.pdf}}\\
\subfloat[][3]{\includegraphics[width=0.35\textwidth, angle=-90]{/home/hroest/srm_clashes/results/True_False_20710_range0to10000_lendist.pdf}}
\subfloat[][4]{\includegraphics[width=0.35\textwidth, angle=-90]{/home/hroest/srm_clashes/results/True_False_20710_range0to10000.pdf}}\\
\subfloat[][5]{\includegraphics[width=0.35\textwidth, angle=-90]{/home/hroest/srm_clashes/results/True_False_20710_range300to1500_lendist.pdf}}
\subfloat[][6]{\includegraphics[width=0.35\textwidth, angle=-90]{/home/hroest/srm_clashes/results/True_False_20710_range300to1500.pdf}}

\label{label1}
\caption{
Searched 2+ parent and 1+ fragment ion against a background of 2+ parent (1+,2+ fragment) and 3+ parent (1+,2+ fragment).
SSRCalcRange was 2 arbitrary units.
}
\end{figure}

\begin{figure}
\subfloat[][1]{\includegraphics[width=0.35\textwidth, angle=-90]{/home/hroest/srm_clashes/results/True_False_40210_range300to1500_lendist.pdf}}
\subfloat[][2]{\includegraphics[width=0.35\textwidth, angle=-90]{/home/hroest/srm_clashes/results/True_False_40210_range300to1500.pdf}}\\
\subfloat[][3]{\includegraphics[width=0.35\textwidth, angle=-90]{/home/hroest/srm_clashes/results/True_False_40710_range0to10000_lendist.pdf}}
\subfloat[][4]{\includegraphics[width=0.35\textwidth, angle=-90]{/home/hroest/srm_clashes/results/True_False_40710_range0to10000.pdf}}\\
\subfloat[][5]{\includegraphics[width=0.35\textwidth, angle=-90]{/home/hroest/srm_clashes/results/True_False_40710_range300to1500_lendist.pdf}}
\subfloat[][6]{\includegraphics[width=0.35\textwidth, angle=-90]{/home/hroest/srm_clashes/results/True_False_40710_range300to1500.pdf}}

\label{label2}
\caption{
Searched 2+ parent and 1+ fragment ion against a background of 2+ parent (1+,2+ fragment) and 3+ parent (1+,2+ fragment).
SSRCalcRange was 4 arbitrary units.
}
\end{figure}

\begin{figure}

\subfloat[][1]{\includegraphics[width=0.35\textwidth, angle=-90]{/home/hroest/srm_clashes/results/True_True_20210_range300to1500_lendist.pdf}}
\subfloat[][2]{\includegraphics[width=0.35\textwidth, angle=-90]{/home/hroest/srm_clashes/results/True_True_20210_range300to1500.pdf}}\\
\subfloat[][3]{\includegraphics[width=0.35\textwidth, angle=-90]{/home/hroest/srm_clashes/results/True_True_20710_range0to10000_lendist.pdf}}
\subfloat[][4]{\includegraphics[width=0.35\textwidth, angle=-90]{/home/hroest/srm_clashes/results/True_True_20710_range0to10000.pdf}}\\
\subfloat[][5]{\includegraphics[width=0.35\textwidth, angle=-90]{/home/hroest/srm_clashes/results/True_True_20710_range300to1500_lendist.pdf}}
\subfloat[][6]{\includegraphics[width=0.35\textwidth, angle=-90]{/home/hroest/srm_clashes/results/True_True_20710_range300to1500.pdf}}\\

\label{label3}
\caption{
Searched 2+ parent and 1+ fragment ion against a background of 2+ parent and 1+ fragment ions.
SSRCalcRange was 2 arbitrary units.
}
\end{figure}

\begin{figure}
\subfloat[][1]{\includegraphics[width=0.35\textwidth, angle=-90]{/home/hroest/srm_clashes/results/True_True_40210_range300to1500_lendist.pdf}}
\subfloat[][2]{\includegraphics[width=0.35\textwidth, angle=-90]{/home/hroest/srm_clashes/results/True_True_40210_range300to1500.pdf}}\\
\subfloat[][3]{\includegraphics[width=0.35\textwidth, angle=-90]{/home/hroest/srm_clashes/results/True_True_40710_range0to10000_lendist.pdf}}
\subfloat[][4]{\includegraphics[width=0.35\textwidth, angle=-90]{/home/hroest/srm_clashes/results/True_True_40710_range0to10000.pdf}}\\
\subfloat[][5]{\includegraphics[width=0.35\textwidth, angle=-90]{/home/hroest/srm_clashes/results/True_True_40710_range300to1500_lendist.pdf}}
\subfloat[][6]{\includegraphics[width=0.35\textwidth, angle=-90]{/home/hroest/srm_clashes/results/True_True_40710_range300to1500.pdf}}\\

\label{label4}
\caption{
Searched 2+ parent and 1+ fragment ion against a background of 2+ parent and 1+ fragment ions.
SSRCalcRange was 4 arbitrary units.
}
\end{figure}


\subsection{Conclusions}
Lars suggests to take the N best peptides (instead of spectra). I will use
more spectra and see what happens.




\section{Yeast}

\subsection{Figures}


For all figures:
Q3 range = 1.0 Da
On the left side, the distribution by peptide length is shown, on the right side the overall distribution of percent unique transitions.
From top to bottom: 
\begin{itemize}
\item Massrange for Q3 300-1500 Da, Q1 window 0.2 Da
\item Massrange for Q3 0-10000  Da, Q1 window 0.7 Da
\item Massrange for Q3 300-1500 Da, Q1 window 0.7 Da
\end{itemize}



\begin{figure}

\subfloat[][1]{\includegraphics[width=0.35\textwidth, angle=-90]{/home/hroest/srm_clashes/results/yeast/yeast_True_False_20210_range300to1500_lendist.pdf}}
\subfloat[][2]{\includegraphics[width=0.35\textwidth, angle=-90]{/home/hroest/srm_clashes/results/yeast/yeast_True_False_20210_range300to1500.pdf}}
%\subfloat[][3]{\includegraphics[width=0.35\textwidth, angle=-90]{/home/hroest/srm_clashes/results/yeast/yeast_True_False_20710_range0to10000_lendist.pdf}}
%\subfloat[][4]{\includegraphics[width=0.35\textwidth, angle=-90]{/home/hroest/srm_clashes/results/yeast/yeast_True_False_20710_range0to10000.pdf}}\\
%\subfloat[][5]{\includegraphics[width=0.35\textwidth, angle=-90]{/home/hroest/srm_clashes/results/yeast/yeast_True_False_20710_range300to1500_lendist.pdf}}
%\subfloat[][6]{\includegraphics[width=0.35\textwidth, angle=-90]{/home/hroest/srm_clashes/results/yeast/yeast_True_False_20710_range300to1500.pdf}}

\label{y_true_false}
\caption{Yeast: Searched 2+ parent and 1+ fragment ion against a background of 2+ parent (1+,2+ fragment) and 3+ parent (1+,2+ fragment).
SSRCalcRange was 2 arbitrary units.
}
\end{figure}

\begin{figure}
\subfloat{\includegraphics{/home/hroest/srm_clashes/results/yeast/q1_q3_hexbins_1200_small_bigfile.png}}
\caption{
\textbf{Two dimensional distribution of SRM-transitions in the tryptic
yeast peptidome (Q1/Q3).}
On the x-axis is the parent ion mass (Q1 mass) and
on the y axis the fragment ion mass (Q3 mass). Doubly and triply charged
parent ions and singly and doubly charged fragment ions were considered.
We see that there are distinct areas in the Q1-Q3 space, some of which are highly
populated and thus might be a bad choice for an SRM measurement. 
The ``streaks'' come from the y\textsubscript{n} ions (the last y ion).
}
%
\label{y_q1q3}

\end{figure}

As can be seen in Figure~\ref{y_q1q3} there are ``streaks'' in the q1/q3
distribution. They are paired since they result from a loss of arginine
(174 Da)
or lysine (146 Da) and we thus see a difference between them of 28 Da
(singly charged fragment ions) and 14 Da respectively. 
From bottom to top we see the lines for 
\begin{itemize}
\item 2+ parent / 2+ fragment
\item 3+ parent / 2+ fragment
\item 2+ parent / 1+ fragment
\item 3+ parent / 1+ fragment
\end{itemize}

\end{document}
